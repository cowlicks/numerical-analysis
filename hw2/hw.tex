\documentclass[a4paper]{article}
\usepackage{amsmath}
\usepackage{listings}
\lstset{breaklines=true}
\title{Numerical Analysis Homework \#2}
\author{Blake Griffith} 
\begin{document}
\maketitle
\begin{section}{Excercises}
	\begin{itemize}
		\item{sec 2.2 1(a)}\\
			I wrote a short python script to do this it is pasted below with the output.
			\lstinputlisting{ex1a_falseposition.py}
			and the output is:
			\begin{lstlisting}
$ python3 ex1a_falseposition.py 
p_0 = 0.8674193917610414 and the interval (0, 1)
p_1 = 0.8842599009093866 and the interval (0.8674193917610414, 1)
p_2 = 0.8845069769792577 and the interval (0.8842599009093866, 1)
p_3 = 0.8845105633505176 and the interval (0.8845069769792577, 1)
p_4 = 0.8845106153993479 and the interval (0.8845105633505176, 1)
p_5 = 0.8845106161547283 and the interval (0.8845106153993479, 1)

			\end{lstlisting}

		\item{sec 2.4 3 (only part (1) and (2))}\\
			Again, I did this with another python script, it is copied below with the output.
			\lstinputlisting{ex3_newton.py}
			and the output is:
			\begin{lstlisting}
$ python3 ex3_newton.py 
part (1)
i_1 = 3.0
i_2 = 2.2
i_3 = 1.8301507537688444
i_4 = 1.7377954531428215
i_5 = 1.7320722915449542

part (2)
|pn - p(n-1)| = 2.0
|p(n-1) - p| = 0.7320508075688772
|pn - p| = 1.2679491924311228

|pn - p(n-1)| = 0.7999999999999998
|p(n-1) - p| = 1.2679491924311228
|pn - p| = 0.467949192431123

|pn - p(n-1)| = 0.3698492462311558
|p(n-1) - p| = 0.467949192431123
|pn - p| = 0.0980999461999672

|pn - p(n-1)| = 0.09235530062602293
|p(n-1) - p| = 0.0980999461999672
|pn - p| = 0.005744645573944274

|pn - p(n-1)| = 0.005723161597867232
|p(n-1) - p| = 0.005744645573944274
|pn - p| = 2.1483976077041333e-05
\end{lstlisting}		
\item{Sec 2.4: 7}\\
	The given function is \(f(x) = x^{2} - a \) so \( f'(x) = 2x\) and according to Newton's method we can say \( g(x) = x - \frac{f(x)}{f'(x)}\)\\
	Plugging in for \(f(x)\) and \(f'(x)\)\\
	\( g(x) = x - (x^{2} - a)/(2x) = x - x/2 -a/2x = (x - a/x)/2 \)\\
	QED
	\end{itemize}
\end{section}
\begin{section}{Programming assignments}
	\begin{enumerate}
		\item
		This was done in python the code is attached below.\\
		\lstinputlisting{p1_bisection.py}
		And the output is:\\
		\begin{lstlisting}
$ python3 p1_bisection.py 
i_1 = 0.75
i_2 = 0.875
i_3 = 0.9375
i_4 = 0.90625
i_5 = 0.890625
i_6 = 0.8828125
i_7 = 0.88671875
i_8 = 0.884765625
i_9 = 0.8837890625
i_10 = 0.88427734375
i_11 = 0.884521484375
i_12 = 0.8843994140625
i_13 = 0.88446044921875
i_14 = 0.884490966796875
i_15 = 0.8845062255859375
i_16 = 0.8845138549804688
i_17 = 0.8845100402832031
i_18 = 0.8845119476318359
i_19 = 0.8845109939575195
\end{lstlisting}

\item
	\lstinputlisting{p2_newton.py}
	and the output
	\begin{lstlisting}
$ python3 p2_newton.py 
i_1 = 1.0
i_2 = 0.8860617363903003
i_3 = 0.8845109403287758
i_4 = 0.8845106161658667
\end{lstlisting}
\end{enumerate}
\end{section}
\end{document}

