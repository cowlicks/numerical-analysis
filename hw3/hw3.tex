\documentclass[a4paper]{article}
\usepackage{amsmath}
\title{Numerical Analysis Homework 3}
\author{Blake Griffith}
\begin{document}
\maketitle
\begin{section}{Excersise}
	\begin{itemize}
		\item{Sec 3.0: 16}\\
			A matrix is singular when it's determinant equals zero.
			\begin{enumerate}
				\item \( |A| = \alpha^2 - 4 = 0 \rightarrow \alpha = \pm 2 \)
				\item \( |B| = -2( \alpha + 15) + \alpha ( 3\alpha + 5) = 0 \rightarrow \alpha = - 1 \pm \sqrt{41} / 2 \)\\
					by the quadratic formula.
			\end{enumerate}
		\item{Sec 3.0: 1, 3}\\ 
			For this I used the algorithm I wrote for the following homework assingments, slightly modified to fit the matrix.
1) a = ([(2, -1, 1, -1),
	(4, 2, 1, 4)
	(6, -4, 2, -2)])
3) b = ([(1, 2, -1, 1)
	(2, -1, 1, 3)
	(-1, 2, 3, 7)])
	\end{itemize}
\end{section}
\end{document}

